\documentclass[a4paper, 12pt]{scrartcl} 
\usepackage{ulem}
\usepackage{tabularx}
\usepackage{graphicx}
\usepackage{amsmath}
\usepackage{lmodern}
\usepackage{cleveref}
\usepackage[warn]{mathtext} 
\usepackage[utf8]{inputenc}
\usepackage[english]{babel} 
\usepackage{indentfirst} 
\usepackage{misccorr} 
\usepackage{caption}
\usepackage{graphicx}
\usepackage{subcaption}
\usepackage{xcolor}
\usepackage{cite}
\usepackage{wrapfig}
\usepackage{setspace}
\usepackage{algorithm}
\usepackage{algpseudocode}
\captionsetup{compatibility=false}
\crefname{subsection}{subsection}{subsections}
\bibliographystyle{unsrt}

\title{The Mammalian Sibelia Project}
\author{Ilia Minkin}
\date{}
\begin{document}
\maketitle
\algnotext{EndFor}
\algnotext{EndIf}
\algnotext{EndWhile}
\renewcommand{\algorithmicrequire}{\textbf{Input: }}
\renewcommand{\algorithmicensure}{\textbf{Output: }}

\section{Problem Formulation}
\textbf{Input:} $S$: Set of $m$ strings of equal length $n$ and two positive integers $k$ and $c$.

\textbf{Output:} $G$: De Bruijn graph of $S$ constructed for substrings of length $k$ with all bulges of size $\leq c$ collapsed into paths.
Each edge of $G$ also has a list of intervals of $S$ associated with it.

\textbf{Question:} Come up with an algorithm with a low time/space complexity, in both $m$ and $n$. 

A note: while current version of Sibelia consequently constructs and collapses several graphs for different values of $k$ and $c$, for sake of simplicity at this stage we assume we are given a single $k$ and a $c$.

The problem essentially consists of two parts: construction of the compressed de Bruijn graph out of $S$ and its modification.
While the graph construction can be well defined, its simplification is trickier.

\section{Graph Construction}

\subsection{Algorithm Description}
The problem of graph construction is as follows:

\textbf{Input:} $S$: Set of $m$ strings of equal length $n$ and a positive integers $k$.

\textbf{Output:} $G = (V, E)$: A compressed de Bruijn graph of $S$ constructed for substrings of length $k$.

Once the vertex set $V$ of the compressed graph is identified, construction of edge set $E$ is straightforward.
Suppose that there is a function $V_{id}(s)$ that returns number of vertex in $G$ labeled with the $k$-mer $s$ or $-1$ if there is no such vertex.
Given $V_{id}$ we construct edge set using the following algorithm:
\begin{algorithm}[H]
\caption{Vertex Set of the Compressed Graph} 
\algorithmicrequire{A collection of string $S = \{s_1, \ldots, s_m\}$, an integer $k$ and function $V_{id}$} \\ 
\algorithmicensure{Edge set $E$ of $G$}
\begin{algorithmic}[1]
\State $G \gets \emptyset$
\For{$s \in S$}
	\State $u \gets V_{id}(s[1:k])$
	\For{$i \gets 2$ \textbf{to} $|s|$}
		\State $v \gets V_{id}(s[i:i + k - 1])$
		\If{$v \neq -1$}
			\State $G \gets G \cup (u, v)$
			\State $u \gets v$
		\EndIf
	\EndFor
\EndFor
\end{algorithmic}
\end{algorithm}

One of the possible ways to construct $V_{id}$ efficiently is to scan the input strings two times:
\begin{enumerate}
\item At the first pass, store all $(k + 1)$-mers in a set $T$ using e.g. a Bloom filter
\item At the second pass, construct the dictionary of vertices by testing at each position if there are different edges leaving or entering the current $k$-mer 
\end{enumerate}
So the algorithm is the following:
\begin{algorithm}[H]
\caption{Edge Set of the Compressed Graph} 
\algorithmicrequire{A collection of string $S = \{s_1, \ldots, s_m\}$, an integer $k$} \\ 
\algorithmicensure{Function $V_{id}$}
\begin{algorithmic}[1]
\State $T \gets \emptyset$
\For{$s \in S$}
	\For{$i \gets 1$ \textbf{to} $|s| - k$}
		\State $T \gets T \cup \{s[i:i + k]\}$
	\EndFor
\EndFor
\State $V_{id} \gets $ an empty dictionary
\For{$s \in S$}
	\For{$i \gets 1$ \textbf{to} $|s| - k + 1$}
		\State $kmer \gets s[i: i + k - 1]$
		\If{$kmer \in V_{id}$}
			\State \textbf{continue}
		\EndIf
		\State $enter \gets 0$ \Comment Number of entering edges
		\State $leave \gets 0$ \Comment Number of leaving edges
		\For{$c \in \{A, C, G, T\}$}
			\If{$kmer + c \in T$}
				\State $leave \gets leave + 1$
			\EndIf
			\If{$c + kmer \in T$}
				\State $enter \gets enter + 1$
			\EndIf
		\EndFor
		\If{$enter > 1$ \text{\textbf{or}} $leave > 1$}
			\State $V_{id}(kmer) \gets |V_{id}|$
		\EndIf
	\EndFor
\EndFor
\end{algorithmic}
\end{algorithm}

A note: the algorithms above consider only positive strands of the input strings, extending them to handle both strands is straightforward.
\subsection{Complexity}
Assuming that we implement the set $T$ as a Bloom filter, we have following for the first stage.
We have $q$ hash functions and a bit array $A$ of size $w$.
Storing all $(k + 1)$-mers requires time $O(nmq)$ and space $O(w + q)$: at each step we update values of $q$ hash functions and modify $A$.
A note: we assume that updating value of each hash function takes constant time (i.e. rolling hash).

The naive implementation of the second stage uses a hash table for $V_{id}$.
Each testing for an edge takes $O(q)$ time since we have to update value of all hash functions to query $T$ and we perform 8 testings for each position.
Insertion to the hash table takes time $O(k)$ and we perform exactly $|V|$ such insertions.

Combining all together we have $O(nmq + k|V|)$ time and $O(|V| + w + q)$ space for both stages. 
It is worth to note that the space complexity depends on the size of output $|V|$ plus size of the bit array $A$ for the Bloom filter.
Size of the bit array only depends on desired false positive rate and can be tuned.
It is important to say that false positives only hurt in the way of adding non-bifurcation vertices to the graph and do not generate any loss of information.
Hence, the trade off between the false positive rate and the filter size could be quite efficient.

Assuming that $G$ is sparse, it takes smaller space than the input itself and is better than using a suffix array (linear space respective to the input) or indexing the whole input in a hash table to store ordinary de Bruijn graph and compress it (a totally naive approach).

An unpleasant fact is that the complexity depends on $k$, unlike the suffix array implementation.
But I believe that we can do much better.

\subsection{Testing Datasets}

\begin{enumerate}
\item A dozen of flies genomes
\item Several tens of vertebrates (mammalians?)
\item Simulated from human chromosome (30-40 genomes)
\end{enumerate}


\pagebreak

\section{Graph Simplification}
A note: during the graph simplification we consider only paths corresponding to some substrings of the input genome.
There is an issue that makes complexity analysis of the simplification tricky: it obviously depends on the structure of the graph, i.e. number of bulges.
One may make an interesting observation: while collapsing a bulge obviously removes it from the graph, it may create other bulges.
So the question is whether this process converges, i.e. do we remove more bulges from the graph than we create.

In practice it seems to converge, but it is possible to draw an example where it will oscillate forever: i.e. removing a bulge $b_1$ creates a bulge $b_2$ and removal of $b_2$ recreates $b_1$ and so on.
It is a very special case that requires presence of at least two copies of substring $s$ such that $s = \, \text{reverse complement of} \, s$.

Another issue is that there is more than one way of simplification of the graph and they yield different graphs.
A possible solution to both of this issues is to propose a concrete strategy of simplification and show that under this strategy:
\begin{itemize}
\item The simplification converges within reasonable time so we can estimate its complexity
\item The resulting graph structure is "nice" in some way of defining niceness
\end{itemize}

\section{Results}
A note: the BF (Bloom Filter) algorithm takes $k$-mers only from the positive strand, I will fix it next time.
\begin{table}[h]

\begin{center}
\caption{Performance Human Chromosome 1}
\begin{tabular}{ccccc}
\hline
Algorithm  & Running time & Memory Usage & Bifurcations Count & False Positives\\
\hline
BF $2^{31}$ bits & 0:29 & 1.35 GB & 1 774 061 & 22 356 788 \\
BF $2^{32}$ bits & 0:22 & 0.77 GB & 1 774 061 & 3 639 291  \\
DSK + BCALM   & 0:07 + 1:56 & 6GB + 20MB & \\
\hline
\end{tabular}

\caption{Performance on 60 E.coli Strains}
\begin{tabular}{ccccc}
\hline
Algorithm  & Running time & Memory Usage & Bifurcations Count & False Positives\\
\hline
BF $2^{31}$ bits & 0:25 & 0.32 GB & 940 002 & 18056 \\
BF $2^{32}$ bits & 0:24 & 0.58 GB & 940 002 & 2477  \\
DSK + BCALM   & 0:06 + 10 &  4GB + 20MB & \\
\hline
\end{tabular}
\end{center}
\end{table}
\pagebreak

Unique $k$-mers and singletons and non-singletons in several vertebrate genomes:
\begin{enumerate}
\item Human
\item Chimpanzee
\item Orangutan
\item Cow
\item Mouse
\item Rat
\end{enumerate}

\begin{table}[h]
\begin{center}
\caption{$k = 25$}
\begin{tabular}{ccccc}
\hline
\#Genomes  & Unique $k$-mers & Non-singletons &  &\\
\hline
\hline
\end{tabular}

\end{center}
\end{table}

\section{Ideas}
\subsection{A-Bruijn Graphs}
Essentially a compressed de Bruijn graph is an A-Bruijn graph defined on the vertex set of bifurcations, see \cite{lin2014manifold} for references.
An idea: maybe we can refine the bifurcations even more or bifurcations for different values of $k$ to make the graph "nicer".
To do: formulate the "nicenecess" and the according bifurcations refinement criteria.
\subsection{A $k$-mer Counter}
We can utilize (k + 1)--mer counter to work with the set of (k + 1)--mers rather than with the genomes themselves.
This way we will be able to filter singletons (at least) or gain some other benefits.
\begin{spacing}{0.9}
\bibliographystyle{plain}
\bibliography{writeup}
\end{spacing}

\end{document}
